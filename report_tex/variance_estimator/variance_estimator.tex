\documentclass[a4paper,10pt]{article}
\usepackage[utf8]{inputenc}
%\usepackage[top=2cm, bottom=2cm, left=2cm, right=2cm]{geometry}
\usepackage{amsfonts}
\usepackage{amsmath}
\usepackage{amssymb}
\usepackage{dsfont}%from the package 'doublestroke'

%%%%%%%%%%%%%%%%%%%%%%%%%%%%%% Document %%%%%%%%%%%%%%%%%%%%%%%%%%%%%%
\title{Variance Estimator \\for Adaptive Multilevel Splitting}
\author{Qiming DU}
\date{}
%\setlength\parindent{0pt}
\begin{document}
\maketitle
\vspace{3cm}
\section*{Introduction}
As we know, the estimation of variance (asymptotic or non-asymptotic)
in the framework
of Adaptive Multilevel Splitting is always problematic. The classic solution is 
to run the algorithm $N$ times independently and to use the empirical 
version of the variance as a crude estimator.
The convergence is then guaranteed by
the Law of Large Numbers. However, as we already know, the cost of the algorithm 
is alway very expensive and this kind of calculation is not always practical. 
An alternative solution which is realtively more efficient is then 
an interesting problem.

Inspired by the variance estimator introduced by Lee \& Whiteley in \cite{lee2015},
which is already prove to be efficient for the particle filter framework,
we would like to prove that the estimator is also usable for the Adaptive 
Multilevel Splitting framework. By using this kind of estimator, all we need is
to run one simulation to give an estimation of the non-asymptotic variance, and
the terms for calculating the asymptotic variance.
The main difference between thess two framework
is that the selection function $G$ and transition kernel $M$ are not
deterministic
for the latter. Thanks to the work by F.Cérou et A.Guyader in \cite{cerou2014fluctuation}, we have the tools to deal with the randomness of $G$ and $M$.    

We remark that the notations used for the following sections are mainly from \cite{lee2015} and \cite{cerou2014fluctuation}.   

\section{Fixed Level Method}
In this seciton, we firstly consider a general Feynman-Kac framework 
where all the selection function $G$ and transition kernel $M$ are deterministic. We will try to give a more intuitive construction of the estimator mentioned in \cite{lee2015}. 

In the classic Feynman-Kac model, for example the variaty of models introduced
in \cite{del2004feynman}, we modelize the particle system by "one abstract
particle", with respect to the empirical measure and the transition of the evolution of the
particle system. For example, the transition of the system from level $k$ to
level $k+1$ is
represented by:

\begin{equation}
                \label{FK_classic}
Q_{k+1}(x,\mathrm{d}y) = G_k(x)M_{k+1}(x,\mathrm{d}y)
\end{equation}

This kind of presentation make the so-called cloning procedure invisible.
In fact, the intrinsic problem is that, wo lost the connection between the normalized measure
$\hat\eta_k$ and $\hat \eta_{k+1}$. Both of these two measures are sampled
directly, so the cloning procedure just hide behind it.

In order to have a more precise representation, or in order to get more
information of the cloning procedure, we will introduce a new
notation to indicate the information the branching procedure of
the particle system.
For $p \in [n-1]$ and for $q \in [N]$, where $N$ is the number of particles, we
define

$$
a_p^q:\ \text{the ancestor in the level $p$ of the $q^{th}$ particle in the
level $p+1$}
$$
$$
a_p:\ \text{the vector of ancestors in level $p$ for particles in level $p+1$}
$$





\bibliographystyle{unsrt}%Used BibTeX style is unsrt
\bibliography{citation}%bibtex file should be names as citation.bib
\end{document}
